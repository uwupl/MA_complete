% -*- TeX -*- -*- DE -*-

\chapter{Motivation}\label{ch:Motivation}
Hier kann viel aus Papern übernommen werden. Einige Industriebeispiele wären ganz cool. --> TODO Beispiele, die zitierfähig sind, finden.\\
Wird aber wahrscheinlich eines der letzten Kapitel, die angegangen werden wird. Aktuell keine hohe Priorität (11. Oktober)\\
\section{Anomaliedetektion in der Industrie}\label{sec:AnomaliedetektionIndustrie}
Hier wird die Relevanz von Anomaliedetektion in der Industrie erläutert.\
Wie bereits erklärt --> Industriebeispiele aus der "Echten Welt" --> Muen etc. fragen für Tips.\
\section{Unüberwachte Anomaliedetektion}\label{sec:UnueberwachteAnomaliedetektion}
Hier wird ausgeführt warum überwachte Methoden an ihre Grenzen stoßen und warum unüberwachte Methoden sinnvoll sind.\
Auch hierfür werden noch Quellen gesucht.\
\section{Laufzeitoptimierung und Ressourcenbeschränktheit}\label{sec:Laufzeitoptimierung}
Erster Abschnitt aus efficientad paper kann hier gut zitiert werden.\\
\textit{Real-world anomaly detection applications frequently put constraints on the computational requirements of a method. \ 
There are cases where detecting an anomaly too late can cause substantial economic damage, such as metal objects in a crop \ 
field entering the interior of a combine harvester. In other cases, even human health is at risk, for example, if a limb of \ 
a machine operator approaches a blade. Furthermore, industrial settings commonly involve strict runtime limits caused by high \ 
production rates [4]. Not adhering to these limits would decrease the production rate of the respective application and thus \ 
its economic via-bility. It is therefore essential to pay attention to the compu-tational and economic cost of anomaly detection \ 
methods to keep them suitable for real-world applications.}
Dann wird ausgeführt, warum eine Laufzeitoptimierung sinnvoll ist und warum es auch heute noch sinnvoll sein kann oder nicht ansders möglich ist, \ 
als auf teure GPUs zu verzichten.\
Beispiele für edge devices --> mobile Roboter, oder Drohnen, die autonom agieren sollen, aber nur begrenzte Ressourcen haben und vor allem \
keine GPUs.\\
Können wir hier den Bogen spannen zu Implementierung auf FPGAs? --> Muen fragen und recherchieren.