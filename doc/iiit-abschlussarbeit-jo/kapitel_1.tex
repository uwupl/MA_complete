% -*- TeX -*- -*- DE -*-

\chapter{Einleitung}\label{ch:einleitung} Die Vorlage \lstinline!iiit-dipl.cls! ist die dritte Version der am IIIT fur
Studien- und Diplomarbeiten verwendeten Vorlage. Die aktuellen Revision wurde um einige packages
erweitert, die in den vielen Arbeiten in der einen oder anderen Form zu finden waren oder sich im
taglichen Gebrauch als hilfreich erwiesen haben. Es sollen hier auch kleine Beispiele gegeben
werden, um ein einheitliches Erscheinungsbild der am IIIT erstellten Arbeiten zu gewahrleisten.
Dabei wird ein gewisses Grundwissen uber \LaTeX{} vorausgesetzt.

In Kapitel \ref{ch:einleitung} werden die technischen Grundlagen wie die eingebundenen packages
sowie die Dateistruktur dieser Beispielarbeit beleuchtet. Kapitel \ref{ch:modellierung} geht naher
auf die von den packages sowie der Klasse zur Verfugung gestellten Umgebungen ein.

\section{Dateistruktur}

Das zentrale Dokument einer Studien-/ Diplomarbeit ist \lstinline!BA_MA_NameOfStudent.tex!. Am Anfang dieser
Datei werden die grundlegenden Einstellungen an der Dokumentenklasse vorgenommen. Diese werden als
Argumente in eckigen Klammern an die Klasse ubergeben.

Die Optionen der Dokumentenklasse konnen beispielsweise folgendermasen eingestellt werden:
\lstinputlisting[style=latex,firstline=2,lastline=31,firstnumber=2]{\jobname.tex} Damit wird die Klasse
\lstinline!iiit-dipl! mit doppelseitigem Layout, alphanumerischem
Literaturverzeichnis ausgewahlt.

Einige der zur Verfugung stehenden switches sind in Tabelle \ref{tab:switches} aufgefuhrt und erlautert.
Defaultwerte sind mit eckigen Klammern markiert.

\newcolumntype{R}{>{\raggedleft\arraybackslash}X}
\begin{table}[tb]
    \centering
    \caption{Zentrale switches\label{tab:switches}}
    \begin{tabularx}{\textwidth}{>{\hsize=.6\hsize}R>{\hsize=1.4\hsize}X}
        \toprule
        switch          & Bedeutung\\
        \midrule
        \lstinline!latin1!, \lstinline!utf8!, \lstinline![ansinew]! & Zur direkten Eingabe von
                          Sonderzeichen wie a, o, u und s muss dem Compiler die verwendete
                          Kodierung mitgeteilt werden. \\
        \lstinline![deutsch]!, \lstinline!english! & Sprache der automatisch gesetzten Begriffe \\
        \lstinline!oneside!, \lstinline![twoside]! & Einstellung fur doppelseitigen Druck.
                          Beeinflusst die Kopfzeile und Leerseiten vor neuen Kapiteln. \\
        \lstinline!biblatex! & Verwendet \lstinline!biblatex! statt \lstinline!bibtex!.
                          \lstinline!biblatex! hat den Vorteil, dass es die UTF8-Kodierung
                          unterstutzt, die beispielsweise von JabRef verwendet wird. Damit konnen
                          beispielsweise auch Referenzen von Autoren mit Umlauten im Namen korrekt
                          sortiert werden. \lstinline!biblatex! erfordert, dass in der verwendeten
                          Umgebung \lstinline!biber.exe! als Bibtex-Compiler eingestellt ist. Falls
                          \lstinline!biber! nicht in der installierten \TeX-Distribution enthalten ist,
                          kann es nachtraglich von
                          \url{http://sourceforge.net/projects/biblatex-biber/files/biblatex-biber/}
                          bezogen werden. Die Kompatibilitaten der verschiedenen \lstinline!biblatex!
                          und \lstinline!biber! Versionen konnen der \lstinline!biblatex! Dokumentation
                          \cite{biblatex} entnommen werden. \\
        \lstinline!lst! & Sollte angegeben werden, wenn Quelltexte eingebunden werden. Mit dieser
                          Option wird das Listings-Package geladen und es werden Styles mit
                          Syntax-Highlighting fur verschiedene Programmiersprachen (\lstinline!C++!,
                          \lstinline!matlab!, \lstinline!latex!, \lstinline!other!) vordefiniert. Wird
                          das Listings-Package nicht benotigt, empfiehlt es sich, diese Option
                          wegzulassen, um die Compiliergeschwindigkeit zu erhohen.\\
        andere          & Andere Optionen werden von der \lstinline!scrbook!-Klasse behandelt.\\
        \bottomrule
    \end{tabularx}
\end{table}

Um die Titelseite korrekt generieren zu konnen, mussen einige (selbsterklarende) Variablen gesetzt
werden. Dabei ergeben die folgenden Einstellungen die Titelseite dieses Dokuments:
\lstinputlisting[style=latex,firstline=35,lastline=41,firstnumber=35]{\jobname.tex}
Danach kann mit dem eigentlichen Dokument begonnen werden. Zuerst wird die frontmatter, also das
vorbereitende Material prasentiert. Hierzu gehoren die Titelseite mit Einverstandniserklarung
(\lstinline!\maketitle!) und das Inhaltsverzeichnis (\lstinline!\tableofcontents!), die automatisch
generiert werden, sowie der Abstract, eine ca. 100-200 Wort lange Zusammenfassung der Arbeit
inklusive der erzielten Ergebnisse auf Englisch.

Im Hauptteil (\lstinline!\mainmatter!) wird die eigentliche Arbeit prasentiert. Der Hauptteil
sollte die folgenden Teile beinhalten:
\begin{itemize}
      \item
            Einleitung
      \item
            Zusammenfassung und Ausblick
\end{itemize}
Es ist sinnvoll, den Hauptteil in mehrere Dateien aufzuteilen. Zusatzliche Dateien konnen mit Hilfe des
\lstinline!\include!-Befehls eingebunden werden. Dabei wird der Inhalt der Datei so interpretiert, als
stunde er direkt in der Hauptdatei.

Der letzte Teil des Hauptteils sind die Anhange. In ihnen werden z.B. Variablen, wichtige Teile des
Quellcodes oder langere Herleitungen beschrieben. Nach dem \lstinline!\appendix!-Befehl werden die
Anhange automatisch alphabetisch nummeriert.

Am Abschluss des Dokuments steht das wichtige Literaturverzeichnis sowie ggf. das Bilder- und
Tabellenverzeichnis.

\lstinputlisting[style=latex,firstline=55,lastline=56,,firstnumber=55]{\jobname.tex}

Da die Vorlage, also \lstinline!iiit-dipl.cls! \textit{nicht} verandert werden soll, steht fur
eigene Erweiterungen wie zum Beispiel neue nutzliche packages die Datei
\lstinline!erweiterungen.tex! zur Verfugung, die in die Hauptdatei eingebunden wird.
\lstinputlisting[style=latex,firstline=32,lastline=32,,firstnumber=32]{\jobname.tex}
In ihr wird standardmasig das Verzeichnis fur die Bilder gesetzt sowie ein Beispiel fur eine
erweiterte Trennregel gegeben. Zweck dieser Datei ist es, alle Anpassungen, die die aktuelle Arbeit
benotigt, zentral zu speichern und damit unabhanig von der eigentlichen Diplomklasse zu machen.

\lstinputlisting[style=latex]{erweiterungen.tex}

\section{Allgemeine Hinweise}

Im Laufe der Zeit haben sich verschiedene Vorgehensweisen bei der Erstellung einer Diplom- bzw.
Studienarbeit als nutzlich erwiesen. Einige von ihnen sollen hier ohne Anspruch auf Vollstandigkeit
aufgefuhrt werden.

\begin{description}
      \item[Fruh dokumentieren]
            Es macht Sinn, auch Zwischenergebnisse ggf. auch in \LaTeX{} zu dokumentieren.

      \item[Sinnvolle Dateigrosen]
            \LaTeX{} bietet die Moglichkeit, die Arbeit auf mehrere Dateien aufzuteilen und diese in die
            Hauptdatei einzufugen.
\end{description}
